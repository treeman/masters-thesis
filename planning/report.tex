
\documentclass[11pt]{article}
\usepackage[utf8]{inputenc}
\usepackage{listingsutf8}
\usepackage{fullpage}
%\usepackage{times}
\usepackage{graphicx}
\usepackage{placeins}
%\usepackage{multicol}
%\usepackage{amsmath}
%\usepackage{tabularx}
%\usepackage{setspace}
%\usepackage{epstopdf}
%\usepackage{amassymb}
%\usepackage{latexsym}
%\usepackage[swedish]{babel}
%\usepackage{datetime}
\usepackage{hyperref}
%\usepackage{listings}
\raggedright

% Space up paragraphs a bit
\setlength{\parskip}{4mm plus1mm minus3mm}

%\linespread{1.3} % 1.5 word linespacing. \linespread{1.6} => double

\begin{document}
%\begin{titlepage}
\begin{center}

%\textsc{\LARGE University of Beer}\\[1.5cm]

{ \Large Master's thesis planning report\\[0.5cm] }

% Title
%\HRule \\[0.4cm]
{ \LARGE Production of a modern, \\scalable and unique recommendation engine \\[0.4cm] }

Jonas Hietala, mail@jonashietala.se, jonhi121 \\[0.2cm]

\today

\end{center}


\section*{Problem description}

\subsection*{Background}

Comordo Technologies is a startup in recommender systems driven inside the bounds of LiU's incubator LEAD in Linköping and will in the future offer a cloud service for e-commerce. The base for the recommender system is algorithms based on predication and machine learning. The company now stands to build a first version of it's recommender system.

Personal recommendations are given to individual users given the user's interaction history. For example given a list of movies the user has purchased the recommender system can recommend other similar movies which the user might find interesting. To produce these recommendations the system examines a dataset which contains a large number of users' interaction history.

The master's thesis consists of developing business logic and control programs so the recommender algorithms are runnable on a systems level. The business logic consists of data handling and control programs for the recommender engine and the control programs handles the learning and optimization of learning parameters. The recommender engine will handle purchase history without any ratings, just user-product interaction pairs.

\newpage


\subsection*{Questions}

\begin{itemize}

    \item How can a recommender system be designed to allow for easily extendible input- and output handling?


    \item How can learning and recommendation using \textit{link-analysis} and \textit{katz-eig} be performed in practice, with regards to speed and recommendation quality?

        \begin{itemize}
            \item How shall learning and optimization of their parameters be done?
        \end{itemize}

        To answer that question, an exploration of the function space of the parameters with regards to the evaluation criteria might be necessary. Then additional questions might need answers:

          \begin{itemize}
                \item Is the function convex or concave?
                \item What search optimization techniques are applicable?
                \item Are there large differences in value across the function space?
                \item In what range is it reasonable to expect the optimal parameters to be found in?
          \end{itemize}


    \item Is clustering beneficial for speed or recommendation quality, with respect to \textit{link-analysis} and \textit{katz-eig}?

          The hypothesis is improvement in both aspects and it may allow handling of larger datasets.

          \begin{itemize}
                \item Is it possible to utilize parallelization in conjunction with clustering?
          \end{itemize}

          Clustering groups related users and items together to produce smaller sets of history data and recommendation data could be generated for them independently. Afterwards they can be combined to generate recommendations for the initial larger dataset.

\end{itemize}


\subsection*{Datasets}

Comordo has access to two datasets from clients and as the project goes on more data may become available.

One dataset has approximately 8300 users, 700 items and 200000 elements in the purchase history.

The other dataset has approximately 75000 users, 350000 items with 2000000 elemens in the purchase history, after filtering away users with less than 2 bought items.

Then there is also the MovieLens dataset: \url{http://grouplens.org/datasets/movielens/} where the MovieLens 1M dataset has 6040 users, 3700 items and 1 million ratings. The problem with this dataset is that the master's thesis is about binary user history, not a history of user ratings. The dataset can still be used by converting all ratings to a value of 1, but then the data is modified and it's hard to evaluate it's usefulness.


\section*{Approach}

The project will use iterative development, try to make the system as a whole running up to a point and then iteratively introduce improvements and new functionality. Focus at the beginning will be on building a first version of a control program for and algorithm and then build the data handling around to produce a first version of the system. Focus then switches over to analyzing and controlling the algorithms.

The planned way to show a satisfactory solution for the system construction part is with a demo of the system with customer delivered data. Recommendation quality should be judged based on Precision, Recall and F-measure using different data sets.

\newpage


\section*{Time plan}

\subsection*{Completed milestones}

\begin{tabular}{l l}
    2015-W5     & Basic control program for an algorithm.  \\
    2015-W6     & Simple input and output modules.  \\
    2015-W8     & Plugin system for input data reading.  \\
    2015-W9     & System backend, with data handling, complete.  \\
    2015-W11    & First version of control programs up and running. \\
\end{tabular}

\subsection*{Planned milestones}

\begin{tabular}{l l}
    2015-W15    & Half time evaluation. Demo system. \\
    2015-W18    & Software implementation complete. \\
    2015-W18    & First thesis draft for supervisor review. \\
    2015-W19    & Second thesis draft. \\
    2015-W20    & Thesis draft for opposition. \\
    2015-W21    & Approval for presentation. \\
    2015-W23    & Presentation. \\
\end{tabular}

\subsection*{Unplanned milestones}

\begin{tabular}{l l}
    Opposition  & Oppose on a fellow student. \\
    Publication  & Submit to LiU E-press after presentation. \\
\end{tabular}

\subsection*{Half time evaluation}

Data handling, complete with input and output handling. Databases integrated. Control programs functional, even though not fully optimized.

Report started and introduction and background $90\%$ complete. Largest and most important parts of theory done.

\newpage


\section*{Litterature base}

\textit{More to be added as work goes on}

\begin{enumerate}

    \item J. Bennett och S. Lanning. ``The Netflix Prize''. \textit{KDD Cup and Workshop} (2007)

    \item Z. Huang, D. Zeng, Hsinchun Chen. ``A Comparison of Collaborative-Filtering Recommendation Algorithms for E-commerce''. (2007)

    \item Y. Hu, Y. Koren, C.  Volinsky. ``Collaborative Filtering for Implicit Feedback Datasets''. (2008)

    \item G. Takacs, I. Pilaszy och B. Nemeth. ``Matrix Factorization and Neighbor Based Algorithms for the Netflix Prize''. (?)

    \item \url{https://class.coursera.org/ml-005/lecture/preview}
    \item \url{https://class.coursera.org/recsys-001/lecture/preview}

    \item \url{http://techblog.netflix.com/2012/04/netflix-recommendations-beyond-5-stars.html}
    \item \url{http://techblog.netflix.com/2012/06/netflix-recommendations-beyond-5-stars.html}
    \item \url{http://techblog.netflix.com/2013/03/system-architectures-for.html}


\end{enumerate}

\newpage

\end{document}


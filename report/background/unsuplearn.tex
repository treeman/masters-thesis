
\subsection{Unsupervised learning}

In contrast with supervised learning, \textit{unsupervised learning} doesn't have an expected output to learn from. Instead the task is to learn patterns in the input without any feedback.  The most common unsupervised learning task is \textit{clustering}: detecting potentially useful clusters, or groups, of input examples \citep{norvigAI}.  

A common clustering technique is \textit{k-means}, which clusters around $k$ clusters \citep{jain1999data}. Another technique is \textit{spectral clustering} which is described in more detail in \sectionref{sec:result:clusters} where it is used to find clusters the datasets.





\subsection{Unsupervised learning}

%\Warning[QUEST]{ Decide if this section is even needed! }

In contrast with supervised learning, \textit{unsupervised learning} doesn't have an expected output to learn from. Instead the task is to learn patterns in the input without any feedback.

The most common unsupervised learning task is \textit{clustering}: detecting potentially useful clusters, or groups, of input examples \citep{norvigAI}.

A common clustering technique is \textit{k-means}, a simple \textit{k-means} algorithm is described in \appendixref{app:kmeans}, which iteratively clusters around $k$ clusters. Another technique is \textit{spectral clustering} which is described in more detail in \sectionref{sec:result:clusters} where it is used to find clusters the datasets.



\subsection{Optimization}\label{sec:background:theory:opt}

Most supervised learning algorithms try to minimize a cost function during the learning phase. This function computes a value given some learned parameters and it can vary with different algorithms. The cost function does not make a comparison between two different sets but it only operates on a single training set.

For example a cost function could be defined as

\begin{equation}
    \min_{x_*, y_*} \sum_{h_{u,i} \text{ is known} } (h_{u, i} - x_{u}^T y_i)^2
\end{equation}

where the optimization objective is $x_u$ and $y_i$. Usually stochastic gradient descent is used to find the parameters \cite{hu2008collaborative}. With regularization a possible cost function could be

\begin{equation}
    \min_{x_*, y_*} \sum_{h_{u,i} \text{ is known} } (h_{u, i} - x_{u}^T y_i)^2 + \lambda(\|x_u\|^2 + \|y_i\|^2)
\end{equation}

where $\lambda$ is the regularization parameter found using model selection. This directly penalizes larger values of $x_u$ and $y_i$ which in this case corresponds to an increase in complexity.

This requires the algorithm to be modeled to use user and item vectors $x_u$ and $y_i$ which combine into the actual recommendation as $r_{u, i} = x_u^T y_i$. In effect a simpler cost function (without regularization) could be defined as

\begin{equation}
    \min_{r_{u, i}} \sum_{h_{u,i} \text{ is known} } (h_{u, i} - r_{u, i})^2
\end{equation}

%\textit{Model selection} is the problem of choosing a set of parameters for a learning algorithm with the goal of optimizing the algorithm's performance on an independent data set.
%\Warning[TODO]{ Reformulate }
%\textit{Hyperparameter optimization}

For algorithms where a cost function isn't applicable and during model selection where comparisions between more than a single training set are made, there are a couple of different techniques used:

\subsubsection{Grid search}

Grid search selects a limited parameter space where it evaluates the function \footnote{Suggested by Andrew Ng, Stanford. \url{https://class.coursera.org/ml-006}}. Easily parallelized but it suffers from the curse of dimensionality.


\subsubsection{Bayesian Optimization}

Noisy black-box functions. Develop a statistical model over the function space and pick samples which trades of exploration and exploitation. Has been shown \citep{snoek2012practical} to give better results with fewer experiments than grid search.


\subsubsection{Random Search}

Grid search is exhaustive and possibly expensive, random search with a fixed limit of samples has been shown to be more effective in high-dimension spaces. \citep{bergstra2012random}


%\subsubsection{Gradient Based Optimization}

%For algorithms where it's possible to find a gradient, specialized optimization techniques can be used.
%\Warning[TODO]{ More? }
%\Warning[TODO]{ Ref }

%\begin{equation} \label{eq:rmse}
    %\mathtt{RMSE} = \sqrt{\mathbf{E}\left\{ (\hat{R} - R)^2 \right\}}
%\end{equation}


\subsection{Recommendation model}\label{sec:background:theory:model}

Given a set of users $U$, a set of items $I$ and an interaction history $h_{u, i}$ given in \textit{unweighted binary form}

\begin{equation}\label{eq:hist}
    h_{u, i} = \begin{cases}
        1 \quad \text{if user $u$ has interacted with item $i$} \\
        0 \quad \text{otherwise}
    \end{cases}
\end{equation}

the \textit{recommender problem} is defined by producing a set of recommendations $r_{u, i}$

\begin{equation}\label{eq:binrec}
    r_{u, i} = \begin{cases}
        1 \quad \text{if item $i$ is recommended to user $u$} \\
        0 \quad \text{otherwise}
    \end{cases}
\end{equation}

to maximize the probability that user $u$ will want to interact with item $i$ in the future, for all users and items.  When $r_{u, i}$ is binary this is a \textit{binary classification} problem. This definition is applicable for \textit{implicit feedback} systems which passively track different sorts of user behaviour. For example link following, interaction time and purchase history.

The recommender problem can be extended to the \textit{Top-N recommender problem} by introducing constraints \eqref{eq:constrain_N} which states that only $N$ recommendations can be presented for each user.

\begin{equation}\label{eq:constrain_N}
    \sum_i r_{u, i} \leq N \quad \forall u
\end{equation}


A variation of the recommender problem is when the interaction history is in \textit{weighted form}, when the values increase with each interaction

\begin{equation}\label{eq:whist}
    h_{u, i} = \begin{cases}
        x \quad \text{user $u$ has interacted $x$ times with item $i$} \\
        0 \quad \text{otherwise}
    \end{cases}
\end{equation}

for example $h_{u, i} = 2$ means that the user $u$ has interacted with item $i$ 2 times. It is possible to allow \textit{implicit feedback} systems to log partial interactions, so $h_{u, i} = 0.7$ could mean that user $u$ has watched 70\% of the movie $i$, in the context of movie watching. \citep{hu2008collaborative}

The converse of \textit{implicit feedback} is \textit{explicit feedback} where the users give direct input regarding their preferences, for example with movie ratings or with likes and dislikes.  Here the definition of the interaction history $h_{u, i}$ is the users' rating history:

\begin{equation}
    h_{u, i} = \begin{cases}
        x \quad \text{the rating $x$ user $u$ gave item $i$} \\
        \emptyset \quad \text{if the user $u$ did not rate item $i$}
    \end{cases}
\end{equation}

\Warning[QUEST]{ Uses h for all equations? }

With ratings $r_{u, i}$ changes to $r_{u, i} = \hat{x}$ where $\hat{x}$ is the rating user $u$ is predicted to give item $i$. This is also a \textit{classification} problem, but the problem changes from assigning a binary value to wanting to predict a rating value.


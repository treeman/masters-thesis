
\chapter{Background}\label{cha:background}

This chapter introduces the mathematical theory behind recommendations and the recommendation model used by this thesis. The recommendation algorithms \textit{katz-eig} and \textit{link-analysis} are presented and a summary of machine learning follow which explains supervised learning and the evaluation metrics used. A section about optimization techniques finishes the chapter.


\section{Recommendation theory}

This section introduces the mathematical theory behind recommendations and it presents the two recommendation algorithms \textit{katz-eig} and \textit{link-analysis}.

This is the basic process of producing recommendations:

\begin{enumerate}
    \item Given an interaction history $h_{u, i}$, $u \in Users$, $i \in Items$ and algorithm specific parameters the recommendation algorithm produce recommendations $p_{u, i}$.
    \item The recommendations $p_{u, i}$, which are real values, are converted to binary recommendations $r_{u, i}$ by selecting the $N$ largest $p_{u, i}$ as $r_{u, i} = 1$.
\end{enumerate}

The process of parameter tuning used is as follows:

\begin{enumerate}
    \item Split the interaction matrix $A$ into a training set $A_{train}$, a validation set $A_{val}$ and a test set $A_{test}$.
    \item Evaluate different parameters by producing recommendations with $A_{train}$ and evaluating them against $A_{val}$ or $A_{test}$ with respect to \textit{F-measure}.
    \item Select the best performing parameters with respect to \textit{F-measure}.
\end{enumerate}


\subsection{Model}\label{sec:background:theory:model}

Given a set of users $U$, a set of items $I$ and an interaction history $h_{u, i}$ given in \textit{unweighted binary form}

\begin{equation}\label{eq:hist}
    h_{u, i} = \begin{cases}
        1 \quad \text{if user $u$ has interacted with item $i$} \\
        0 \quad \text{otherwise}
    \end{cases}
\end{equation}

the \textit{recommender problem} is defined by producing a set of recommendations $r_{u, i}$

\begin{equation}
    r_{u, i} = \begin{cases}
        1 \quad \text{if item $i$ is recommended to user $u$} \\
        0 \quad \text{otherwise}
    \end{cases}
\end{equation}

to maximize the probability that user $u$ will want to interact with item $i$ in the future, for all users and items.  This definition is applicable for \textit{implicit feedback} systems which passively track different sorts of user behaviour. For example link following, interaction time and purchase history.

The recommender problem can be extended to the \textit{Top-N recommender problem} by introducing constraints \eqref{eq:constrain_N} which states that only $N$ recommendations can be presented for each user.

\begin{equation}\label{eq:constrain_N}
    \sum_i r_{u, i} \leq N \quad \forall u
\end{equation}


A variation of the recommender problem is when the interaction history is in \textit{weighted form}, when the values increase with each interaction

\begin{equation}\label{eq:whist}
    h_{u, i} = \begin{cases}
        x \quad \text{user $u$ has interacted $x$ times with item $i$} \\
        0 \quad \text{otherwise}
    \end{cases}
\end{equation}

for example $h_{u, i} = 2$ means that the user $u$ has interacted with item $i$ 2 times. It is possible to allow \textit{implicit feedback} systems to log partial interactions, so $h_{u, i} = 0.7$ could mean that user $u$ has watched 70\% of the movie $i$, in the context of movie watching. \citep{hu2008collaborative}

The converse of \textit{implicit feedback} is \textit{explicit feedback} where the users give direct input regarding their preferences, for example with movie ratings or with likes and dislikes.  Here the definition of the interaction history $h_{u, i}$ is the users' rating history.

\begin{equation}
    h_{u, i} = \begin{cases}
        R \quad \text{the rating $R$ user $u$ gave item $i$} \\
        \emptyset \quad \text{if the user $u$ did not rate item $i$}
    \end{cases}
\end{equation}

\Warning[TODO]{ Uses h for all equations? }

The algorithms which produce recommendations produce predictions for each user-item pair, denoted $p_{u, i}$. For interaction history in unweighted binary form if $p_{u, i}$ is close to 1 it means item $i$ is predicted with high probability to user $u$ and a value close to 0 means it's not predicted. If the interaction history instead describes ratings the value corresponds to the predicted ratings the users would give, for example $p_{u, i} = 3.8$ means the algorithm is predicting user $u$ to rate item $i$ a 4, given ratings between 1 and 5.
\Warning[TODO]{ Move to own theory? }

To produce top-N recommendations take the $N$ largest values of $p_{u, i}$ for each user. If it's important to recognize non-recommendations it is possible to set $r_{u, i} = 0$ if $p_{u, i} \leq \epsilon$, for some $\epsilon$, to accommodate for fewer than $N$ recommendations.

This thesis will use an interaction history given in unweighted binary form and will produce recommendations for the Top-N recommender problem.
\Warning[TODO]{ Move to own theory? }

To transform datasets with the more common explicit feedback style of ratings to an unweighted binary form a crude model \eqref{eq:rating2binary} will be used.
\Warning[TODO]{ Move to my own theory?}

\begin{equation} \label{eq:rating2binary}
    h_{u, i} = \begin{cases}
        1 \quad \text{user $u$ has rated item $i$} \\
        0 \quad \text{otherwise}
    \end{cases}
\end{equation}



\subsection{Recommendation prediction}\label{sec:background:theory:pred}

The algorithms which produce binary classification recommendations produce predictions for each user-item pair, denoted $p_{u, i}$. Generally the higher the value of $p_{u, i}$ the more likely is it that user $u$ will interact with item $i$. The predictions $p_{u, i}$ corresponds to the prediction matrix $P = (p_{u, i})$.

$p_{u, i}$ forms the bases for the recommendation set $r_{u, i}$. To produce Top-$N$ recommendations take the $N$ largest $p_{u_k, i} \, \forall i$ for each user $u_k$ and set $r_{u_k, i} = 1$ for these $N$ values. Set $r_{u_k, i} = 0$ for the rest. It is possible to set $r_{u, i} = 0$ if $p_{u, i} \leq \epsilon$, for some $\epsilon$, to accommodate for fewer than $N$ recommendations.

In a classification context when the interaction history describes ratings the value corresponds to the predicted ratings user $u$ would give $i$. The recommendations $r_{u, i}$ then becomes the closest discrete rating value of $p_{u, i}$. For example $p_{u, i} = 3.8$ means a user $u$ is predicted to rate item $i$ a 4, so $r_{u, i} = 4$, given discrete ratings between 1 and 5.

Some algorithms also output a confidence value $c_{u, i}$ which denotes how certain the predicted values are. This is relevant when predicting ratings, for example $p_{u, i} = 4.0$ may seem like a surely predicted 4 rating but a low value of $c_{u, i}$ means we might not want to recommend that item anyway.



\subsection{katz-eig}

The strategies for optimizing algorithm parameters for different datasets, or parameter tuning, has focuses on keeping $\beta$ fixed and searching for $K$ which optimizes \textit{F-measure}. The analysis in \sectionref{sec:param:katzeig} shows that varying $\beta$ has a neglible effect on the algorithm's performance, so $\beta$ is fixed as $\beta = \frac{1}{\|A_{train}\|_2}$.

What follows is a description of the different optimization strategies evaluated:

\begin{description}
    \item[grid]
        Does a grid search over $K$, with a fixed step size. $\beta$ is fixed. $1 \leq K \leq 50$ is examined and a step size of 1 is used.
    \item[rand]
        A random sampling over a subspace of $K$ with a fixed $\beta$. Depends on the size of the subset to sample over and the number of samples. $1 \leq K \leq 50$ is examined and 12 random samples are used meaning 24\% of the subspace is examined.
    \item[hill]
        Hill climbing algorithm using steepest descent which examines the neighbours of $K$ and moves to the best neighbour, will find a local optima. Uses a fixed $\beta$ and a step size of 1 is used.
    \item[stoch-hill]
        An extension of the hill climbing algorithm which does a random restart whenever a local optima is found. Also randomly jumps to a random $K$ by a 10\% probability whenever a step is taken. Depends on the random jump probability, the subspace of $K$ and the number of iterations. A step size of 1 is used, the space restriction is $1 \leq K \leq 50$ and 12 samples are used meaning 24\% of the subspace is examined. The algorithm never revisits an old value.
\end{description}

The following plots compares the recommendation quality and runtime of the different optimization strategies.

\begin{figure}[h!]
    \centering
    \includegraphics[width=0.9\textwidth]{fig/comp/comp_katz_quality.png}
    \caption{Comparison of the recommendation quality given from the parameters found by the different optimization strategies for \textit{katz-eig}.}
    \label{fig:comp_katz_quality}
\end{figure}

\FloatBarrier

\Figureref{fig:comp_katz_quality} show that all the evaluated strategies generate recommendations of a similar quality. The randomized algorithms \textbf{rand} and \textbf{stoch-hill} generally generate slightly worse recommendations due to the variance.

\begin{figure}[h!]
    \centering
    \includegraphics[width=0.9\textwidth]{fig/comp/comp_katz_speed.png}
    \caption{Comparison of the runtime of the different optimization strategies for \textit{katz-eig}, given the optimized parameters specified in \appendixref{app:opt_params}.}
\end{figure}

\begin{figure}[h!]
    \centering
    \includegraphics[width=0.9\textwidth]{fig/comp/comp_katz_speed_log.png}
    \caption{Comparison of the runtime of the different optimization strategies for \textit{katz-eig}, given the optimized parameters specified in \appendixref{app:opt_params}. In a log scale.}
\end{figure}

The runtime is very poor for the grid based approach as expected. The random algorithms also have poor performance, but this could be corrected by reducing the maximum number of samples they try. Reducing them might reduce the recommendation quality which already is worse than that of the regular hill climbing algorithm.

These tests show that the regular hill climbing algorithm is the best optimization strategy for \textit{katz-eig} producing similar recommendation quality to the full grid search while being much faster than the other alternatives.


\newpage

\subsection{link-analysis}

There are two parameters to \textit{link-analysis}: $\gamma$ and $\eta$. They are both continuous and both are undefined at 0.

The following plot is evaluated using \textit{F-measure} w.r.t. the test set using top-10 recommendations over the parameter space of both $\gamma$ and $\eta$.

\begin{figure}[h!]
    \includegraphics[width=\textwidth]{fig/link_eta_gamma/alphaS_link_eta.png}
    \caption{\textit{alphaS}}
    \label{fig:linkalphaS}
\end{figure}

As noted before, the function is undefined at $\gamma = 0$ and $\eta = 0$. For \textit{alphaS} it appears $\gamma < 0$ is a very bad choice. Both $\eta > 0$ and $\eta < 0$ seems to be fair choices.

\begin{figure}[h!]
\centering
\begin{minipage}{.5\textwidth}
    \centering
    \includegraphics[width=\linewidth]{fig/link_eta_gamma/movielens_link_eta.png}
    \captionof{figure}{\textit{movielens1m}}
\end{minipage}%
\begin{minipage}{.5\textwidth}
    \centering
    \includegraphics[width=\linewidth]{fig/link_eta_gamma/romeo_link_eta.png}
    \captionof{figure}{\textit{romeo}}
\end{minipage}
\end{figure}

Similar observations can be made for \textit{movielens1m} and \textit{romeo}. The function space seems to be fairly smooth, almost convex. \textit{alphaS} has an anomaly at $\eta = -2$ and the parameters at 0 are exceptions.

\FloatBarrier

\begin{figure}[h!]
    \includegraphics[width=\textwidth]{fig/link_eta_gamma/eswc2015books.png}
    \caption{\textit{eswc2015books}}
\end{figure}

\FloatBarrier

A closer look at \textit{eswc2015books}, ignoring $\gamma < 0$, reveals that the function space isn't as smooth as it might have seemed in the previous plots, several local optima can be seen. An anomaly around $\eta = -2$ is visible here as well.

\newpage

Interestingly some of the datasets have their maximum at $\eta > 0$ but others have $\eta < 0$. What follows is some plots over $\eta$ and a fixed $\gamma = 2$.

\FloatBarrier

\begin{figure}[h!]
\centering
\begin{minipage}{.5\textwidth}
    \centering
    \includegraphics[width=\linewidth]{fig/link_eta/alphaS_link_eta.png}
    \captionof{figure}{\textit{alphaS}}
\end{minipage}%
\begin{minipage}{.5\textwidth}
    \centering
    \includegraphics[width=\linewidth]{fig/link_eta/eswc2015books_link_eta.png}
    \captionof{figure}{\textit{eswc2015books}}
\end{minipage}
\end{figure}

\begin{figure}[h!]
\centering
\begin{minipage}{.5\textwidth}
    \centering
    \includegraphics[width=\linewidth]{fig/link_eta/movielens_link_eta.png}
    \captionof{figure}{\textit{movielens1m}}
\end{minipage}%
\begin{minipage}{.5\textwidth}
    \centering
    \includegraphics[width=\linewidth]{fig/link_eta/romeo_link_eta.png}
    \captionof{figure}{\textit{romeo}}
\end{minipage}
\end{figure}

\FloatBarrier

Here \textit{alphaS} and \textit{eswc2015books} have maximum \textit{F-measure} with $\eta < 0$ but \textit{movielens1m} and \textit{romeo} with $\eta > 0$.

The actual optima, given optimization criteria of $\eta$ and $\gamma$, can be different. As can be seen in \figureref{fig:linkalphaS} the optima uses a smaller $\gamma$ and a positive $\eta$. But the local optima can still be found with either $\eta > 0$ or $\eta < 0$.

\newpage

$\gamma$ is the parameter which seems to vary the function value the most. The following plots varies $\gamma$ while holding $\eta = 1$ constant.

\FloatBarrier

\begin{figure}[h!]
\centering
\begin{minipage}{.5\textwidth}
    \centering
    \includegraphics[width=\linewidth]{fig/link_gamma/alphaS_link_gamma.png}
    \captionof{figure}{\textit{alphaS}}
\end{minipage}%
\begin{minipage}{.5\textwidth}
    \centering
    \includegraphics[width=\linewidth]{fig/link_gamma/eswc2015books_link_gamma.png}
    \captionof{figure}{\textit{eswc2015books}}
\end{minipage}
\end{figure}

\begin{figure}[h!]
\centering
\begin{minipage}{.5\textwidth}
    \centering
    \includegraphics[width=\linewidth]{fig/link_gamma/movielens_link_gamma.png}
    \captionof{figure}{\textit{movielens1m}}
\end{minipage}%
\begin{minipage}{.5\textwidth}
    \centering
    \includegraphics[width=\linewidth]{fig/link_gamma/romeo_link_gamma.png}
    \captionof{figure}{\textit{romeo}}
\end{minipage}
\end{figure}

\FloatBarrier

\newpage

\section{Machine learning}

In this section a summary of supervised learning explaining how learning from the datasets is accomplished. A short summary of unsupervised learning, mainly focused on clustering, follows and metrics for evaluating recommendation quality is presented at the end of the section.


\subsection{Supervised learning}\label{sec:background:theory:suplearn}

The task of \textit{supervised learning} is given a \textit{training set} with input-output pairs discover a function, the hypothesis, which approximates the input-output mapping.  To measure the accuracy of the hypothesis match it against a \textit{test set} with input-output pairs distinct from the training set.
\citep{norvigAI}

There can be multiple available models for the hypothesis, for example if the hypothesis is a polynomial function of the form 

\begin{equation}
f(x) = a_n x^n + a_{n - 1} x^{n - 1} + ... + a_2 x^2 + a_1 x + a_0
\end{equation}

then the polynomial degree $n = 1, 2, 3, ...$ represents different possible models for the hypothesis \citep{norvigAI}. Other examples include the number of layers and the number of units in a neural network \footnote{Machine Learning, Stanford. \url{https://class.coursera.org/ml-006}} or the rank of a low rank approximation \footnote{katz-eig models this way, see \sectionref{sec:background:theory:katzeig}}.

The different models represents the complexity of the hypothesis. A more complex model can make a better fit to the training data but that introduces the problem of \textit{overfitting} where the hypothesis fits the training data \textit{too} well and it will not fit the test data.
\citep{norvigAI}

\textit{Model selection} is thus performed by evaluating performance with a \textit{validation set}. The reason not to both choose the model and evaluate the model using the test set is that then we will have overfit the test set as we both choose the best model and then evaluate with \textit{the already best fit}. \citep{norvigAI}

The recommended ratio to split the training, validation and test set differs but common recommendations include 60/20/20, 80/10/10, or 70/15/15 \footnote{As recommended by Andrew Ng, Stanford. \url{https://class.coursera.org/ml-006}} depending on domain and the size of the available data set. It is important that the sets are pairwise disjunkt.

If there is no need for a validation set, which can be the case if there are no models to choose from, common training/test set ratios include 70/30, 80/20 or 90/10 \cite{hu2008collaborative, norvigAI} \footnote{Andrew Ng also mentions these values}.

In summary machine learning for supervised learning is done in a couple of steps:

\begin{description}
    \item[Preface] Split data set into training, test and validation sets.
    \item[Training phase] Train the hypothesis using the training set.
    \item[Model selection] Select model using the validation set. (Optional)
    \item[Evaluation] Estimate the accuracy using the test set.
    \item[Application] Apply the developed model to real world data and get results.
\end{description}

Another way to combat overfitting is with \textit{regularization}. Regularization searches for a hypothesis which directly penalizes complexity.  Regularization still needs to select the hyperparameter $\lambda$ using model selection.
\citep{norvigAI}


\subsection{Unsupervised learning}

%\Warning[QUEST]{ Decide if this section is even needed! }

In contrast with supervised learning, \textit{unsupervised learning} doesn't have an expected output to learn from. Instead the task is to learn patterns in the input without any feedback.

The most common unsupervised learning task is \textit{clustering}: detecting potentially useful clusters, or groups, of input examples \citep{norvigAI}.

A common clustering technique is \textit{k-means}, a simple \textit{k-means} algorithm is described in \appendixref{app:kmeans}, which iteratively clusters around $k$ clusters. Another technique is \textit{spectral clustering} which is described in more detail in \sectionref{sec:result:clusters} where it is used to find clusters the datasets.



\subsection{Evaluation}\label{sec:background:theory:eval}

A common technique to evaluate the quality of recommendations as sets is with \textit{Precision}, \textit{Recall} and \textit{F-measure}
\footnote{The 2nd Linked Open Data-enabled Recommender Systems Challenge uses \textit{F-measure}, 2015. \url{http://sisinflab.poliba.it/events/lod-recsys-challenge-2015/}}
. \citep{bobadilla2013recommender}

\Warning[TODO]{ Here $h_{u, i}$ isn't the interaction history, but the test/validation set }

First, \textit{true positives}, $TP$, is the sum of all correctly predicted positive samples.

\begin{equation} \label{eq:tp}
    TP = \sum_{u, i} r_{u, i} = 1 \land \, h_{u, i} = 1
\end{equation}

Conversely \textit{false positives}, $FP$, is the sum of all falsely predicted positive samples.

\begin{equation} \label{eq:fp}
    FP = \sum_{u, i} r_{u, i} = 1 \land \, h_{u, i} = 0
\end{equation}

And \textit{false negatives}, $FN$, is the sum of all falsely predicted negative samples.

\begin{equation} \label{eq:fn}
    FN = \sum_{u, i} r_{u, i} = 0 \land \, h_{u, i} = 1
\end{equation}

Then \textit{Precision} $P$ and \textit{Recall} $R$ is defined as

\begin{equation} \label{eq:precision}
    P = \TP / (\TP + \FP)
\end{equation}

\begin{equation} \label{eq:recall}
    R = \TP / (\TP + \FN)
\end{equation}

Loosely \textit{Precision} signifies how well the recommended items correspond to the users' preferences and \textit{Recall} signifies how well the users' preferences fits with the recommendations.

In many ways precision and recall are competing measures, when optimizing for precision recall decreases and vice versa.  As the number of recommendations $N$ grow precision is expected to be lower and recall is expected to be higher. \citep{hu2008collaborative}


\textit{F-measure}, $\mathit{F1}$, is defined as the harmonic mean of precision and recall \eqref{eq:f1} as a combined measure of precision and recall.

\begin{equation} \label{eq:f1}
    \mathit{F1} = 2 * P * R / (P + R)
\end{equation}

%\Warning[TODO]{ Evaluations for top-N }

% TODO remove?
Another evaluation method commonly used to evaluate explicit feedback system with ratings is the Root of Mean Square Error \rmse \hspace{0.2ex} \eqref{eq:rmse} where $\mathit{pval}$ is the predicted ratings for the user-item interaction matrix $h$.
\citep{bobadilla2013recommender}
%\Warning[TODO]{ Conflicting variable names? }

\begin{equation} \label{eq:rmse}
    \mathtt{RMSE} = \sqrt{\mathbf{E}\left\{ (\mathit{pval} - h)^2 \right\}}
\end{equation}
\Warning[TODO]{ Matrices are usually big lettered! }



\newpage


\subsection{Optimization}\label{sec:background:theory:opt}

Most supervised learning algorithms try to minimize a cost function during the learning phase. This function computes a value given some learned parameters and it can vary with different algorithms. The cost function does not make a comparison between two different sets but it only operates on a single training set.

For example a cost function could be defined as

\begin{equation}
    \min_{x_*, y_*} \sum_{h_{u,i} \text{ is known} } (h_{u, i} - x_{u}^T y_i)^2
\end{equation}

where the optimization objective is $x_u$ and $y_i$. Usually stochastic gradient descent is used to find the parameters \cite{hu2008collaborative}. With regularization a possible cost function could be

\begin{equation}
    \min_{x_*, y_*} \sum_{h_{u,i} \text{ is known} } (h_{u, i} - x_{u}^T y_i)^2 + \lambda(\|x_u\|^2 + \|y_i\|^2)
\end{equation}

where $\lambda$ is the regularization hyperparameter found using model selection. This directly penalizes larger values of $x_u$ and $y_i$ which in this case corresponds to an increase in complexity.

This requires the algorithm to be modeled to use user and item vectors $x_u$ and $y_i$ which combine into the actual recommendation as $r_{u, i} = x_u^T y_i$. In effect a simpler cost function (without regularization) could be defined as

\begin{equation}
    \min_{r_{u, i}} \sum_{h_{u,i} \text{ is known} } (h_{u, i} - r_{u, i})^2
\end{equation}

%\textit{Model selection} is the problem of choosing a set of parameters for a learning algorithm with the goal of optimizing the algorithm's performance on an independent data set.
%\Warning[TODO]{ Reformulate }
%\textit{Hyperparameter optimization}

For algorithms which do not produce parameters from which recommendations and a suitable cost function can be calculated but rather produce the recommendations themselves other objectives can be used....

For algorithms where a cost function isn't applicable and during model selection where comparisons between more than a single training set are made, there are a couple of different techniques used:

\textit{Rewrite!!!!}
\Warning[TODO]{ Rewrite!! }

\subsubsection{Grid search}

Grid search selects a limited parameter space where it evaluates the function \footnote{Suggested by Andrew Ng, Stanford. \url{https://class.coursera.org/ml-006}}. Easily parallelized but it suffers from the curse of dimensionality.


\subsubsection{Bayesian Optimization}

Noisy black-box functions. Develop a statistical model over the function space and pick samples which trades of exploration and exploitation. Has been shown \citep{snoek2012practical} to give better results with fewer experiments than grid search.


\subsubsection{Random Search}

Grid search is exhaustive and possibly expensive, random search with a fixed limit of samples has been shown to be more effective in high-dimension spaces. \citep{bergstra2012random}


%\subsubsection{Gradient Based Optimization}

%For algorithms where it's possible to find a gradient, specialized optimization techniques can be used.
%\Warning[TODO]{ More? }
%\Warning[TODO]{ Ref }

%\begin{equation} \label{eq:rmse}
    %\mathtt{RMSE} = \sqrt{\mathbf{E}\left\{ (\hat{R} - R)^2 \right\}}
%\end{equation}



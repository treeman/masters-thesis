
\newpage
\section{Evaluation}\label{sec:theory:eval}

Binary classifications are commonly evaluated using \textit{Precision}, \textit{Recall} and \textit{F-measure}.
\Warning[TODO]{ Ref needed }
\footnote{eswc competition uses \textit{F-measure}}

First, \textit{true positives}, $TP$, is the sum of all correctly predicted positive samples.

\begin{equation} \label{eq:tp}
    TP = \sum_{u, i} r_{u, i} = 1 \land \, h_{u, i} = 1
\end{equation}

Conversely \textit{false positives}, $FP$, is the sum of all falsely predicted positive samples.

\begin{equation} \label{eq:fp}
    FP = \sum_{u, i} r_{u, i} = 1 \land \, h_{u, i} = 0
\end{equation}

And \textit{false negatives}, $FN$, is the sum of all falsely predicted negative samples.

\begin{equation} \label{eq:fn}
    FN = \sum_{u, i} r_{u, i} = 0 \land \, h_{u, i} = 1
\end{equation}

Using \eqref{eq:tp}, \eqref{eq:fp} and \eqref{eq:fn} \textit{Precision} $P$ and \textit{Recall} $R$ is defined as

\begin{equation} \label{eq:precision}
    P = \TP / (\TP + \FP)
\end{equation}

\begin{equation} \label{eq:recall}
    R = \TP / (\TP + \FN)
\end{equation}

Loosely precision signifies how well the recommended items correspond to the users' preferences and recall signifies how well the users' preferences fits with the recommendations.  

In many ways precision and recall are competing measures, when optimizing for precision recall decreases and vice versa.  As the number of recommendations $K$ grow precision is expected to be lower and recall is expected to be higher. \citep{hu2008collaborative}


\textit{F-measure}, $\mathit{F1}$, is defined as the harmonic mean of precision and recall \eqref{eq:f1} as a combined measure of precision and recall.

\begin{equation} \label{eq:f1}
    \mathit{F1} = 2 * P * R / (P + R)
\end{equation}

\Warning[TODO]{ Evaluations for top-N }

% TODO remove?
%Another evaluation method commonly used to evaluate explicit feedback system with ratings is the root-mean-square error \rmse \hspace{0.2ex} \eqref{eq:rmse} where $\hat{R}$ is the predicted ratings for the user-item matrix $R$.
%\Warning[TODO]{ Conflicting variable names? }

%\begin{equation} \label{eq:rmse}
    %\mathtt{RMSE} = \sqrt{\mathbf{E}\left\{ (\hat{R} - R)^2 \right\}}
%\end{equation}


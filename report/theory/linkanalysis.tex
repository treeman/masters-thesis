\section{The Link-Analysis algorithm}\label{sec:linkanalysis}

\textit{Cleanup, mostly description from article. Reduce information?}

The link-analysis algorithm, as presented by \cite{huang2004link} and further discussed in \cite{huang2007comparison}. See the articles for more information.

The algorithm is an adaptation of HITS \cite{kleinberg1999authoritative} which is a web page ranking algorithm to the recommendation domain. The original algorithm distinguish between \textit{Authoritative} pages which definitely contain high-quality information and \textit{Hub} pages which are comprehensive lists of links to authoritative pages. \citep{huang2007comparison}

The adaptation to the recommendation domain is achieved by introducing the \textit{product representativeness} score $\PR$ and the \textit{consumer representativeness} score $\CR$.

The \textit{product representativeness} score $\PR(i, u)$ can be seen as a measure of the item $i$'s level of interest with respect to user $u$, or in other words $i$'s authority of $u$'s interests in $i$.

The \textit{consumer representativeness} score $\CR(u, \hat{u})$ measures how well $u$ as a hub for $\hat{u}$ associates with products of interests to $\hat{u}$.

If $A$ is the user-item interaction history, as defined by \ref{eq:hist}, 
\Warning[TODO]{ Using $A$ here, definition uses $h$! }
then a recursive definition of the authority and hub scores can be defined as

\begin{equation}
    \PR = A' * \CR
\end{equation}

\begin{equation}
    \CR = A * \PR
\end{equation}

There are two problems with such a definition:
\Warning[TODO]{ This is irrelevant here, could skip? Refer to article instead. }

If a customer has links to all products, that costumer will have the highest representativeness cores for all customers, which is quite useless.  Secondly $\PR$ and $\CR$ will converge to matrices with identical columns.  Therefore $\CR$ is redefined as

\begin{equation}
    \CR = B * \PR + \CR_0
\end{equation}

Where $B$ is a matrix such that $b_{i, j}$:

\begin{equation}
    b_{i, j} = \frac{ a_{i, j} }{ \left(\sum_{j} a_{i, j}\right)^\gamma }
\end{equation}

Meaning $B$ normalizes the representativeness score a costumer receives from linked products by dividing it with the total number of products the customer is linked to.  $\gamma$ controls the extent to which a consumer is penalized for making many purchases.

$\CR_0$ is defined as

\begin{equation}
    cr_{i, j}^0 = \begin{cases}
        \eta \quad \text{if } \; i = j \\
        0    \quad \text{otherwise}
    \end{cases}
\end{equation}

in other words $\CR_0 = \eta * I_M$ where $I_M$ is an $M x M$ identity matrix. It is included to maintain the high representativeness score for the target users themselves. This also necessitates a normalization step to keep the values on a consistent level.

In summary the link-analysis algorithm follow these steps:

\begin{enumerate}
    \item Construct the interaction matrix $A$ and the associating matrix $B$.

    \item Set $\CR_0 = \eta * I_M$ and let $\CR(0) = \CR_0$.
    \item At each iteration $t = 1, \ldots$ perform:

        \begin{enumerate}
            \item $\PR(t) = A' * \CR(t - 1)$
            \item $\CR(t) = B * \PR(t)$
            \item Normalize $\CR(t)$ so each column adds up to 1
                \textit{describe how?}
            \item $\CR(t) = \CR(t) + \CR_0$
        \end{enumerate}

        Repeat until convergence.

\end{enumerate}

There are two parameters to the algorithm: $\gamma$ and $\eta$.
\Warning[TODO]{ Describe them, what's their purpose }


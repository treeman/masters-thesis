\section{The katz-eig algorithm}\label{sec:katzeig}

The original algorithm \cite{katz1953new} is a link prediction algorithm.
\Warning[TODO]{ What does that mean? }
This thesis is using the unweighted definition of \cite{liben2007link} where $h_{u,i} = 1$ if user $u$ has rated item $i$.
\Warning[TODO]{ $A$ vs $h$, update! }


\begin{equation}
    \mathit{pval} = \sum_{t=1}^{\infty} \beta^t A^t
\end{equation}

Here $A^t$ can be seen as the $t$:th link value and $\beta$ signifies the diminishing return for links further away.  As an additional modification the algorithm doesn't operate directly on $A$, but operates on a $k$ rank approximation of $A$.

% Originally introduced by \cite{katz1953new}, described and used by \cite{liben2007link}.

Concretely the katz-eig algorithm follow these steps:

\begin{enumerate}
    \item Construct $U, S, V$ so $U * S * V'$ forms a rank $k$ approximation of $A$, the user-item interaction matrix. Let $S_0 = S$.

    \item At each iteration $t = 1, \ldots, t_{max}$ perform:

        \begin{enumerate}
            \item $S_t = S_{t - 1} + \beta^t * S_{t - 1}^t$
        \end{enumerate}

        Repeat until convergence.

    \item Predicted user-item interaction is given by $\mathit{pval} = U * S_{t_{max}} * V'$.

\end{enumerate}

There are two parameters to the algorithm: $\beta$ and $k$.


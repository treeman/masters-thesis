\chapter{Discussion}\label{cha:discussion}

\textit{Can free form sections here}

\section{Recommender systems}
\textit{Overall design of recommender systems and the one I designed}

The recommender system developed in this thesis is bare bones with not many features. It's reasonable as it's just a first prototype, or a core, for a larger recommender system to be built upon.

The list of features a modern recommender system could have is large. An important feature is the act of explaining recommendations which can be seen in both Netflix and Spotify. According to Netflix
\footnote{\textit{Ref here}}
explaining recommendations improves the effectiveness of the recommender system a lot. This is also supported by the literature
\Warning[TODO]{ Ref }. The reasoning is that it increases the trust for the recommendations, if the recommendations come with a note ``Recommended to you because you watched Die Hard'' and if you liked Die Hard you're more likely to take the recommendations seriously.

Another important feature is diversity of the recommendations. Netflixed notes
\footnote{\textit{ref here}}
that one user account might represent several different people. So a family could share an account at Netflix and use it to watch a variety of different movies, the kids might watch animated movies but the father in the house might watch action movies and someone else might only care for drama. In such cases it's very important not to only recommend one kind of movie but rather recommend different kinds of movies. For example a list of only action movies is not as useful as a mix of action, drama and animated movies.

\textit{Diversity}
\Warning[TODO]{ Ref }
is a metric analogous to \textit{F-measure} but it instead values diversified recommendations. Other recommender algorithms (or a combinations of algorithms) might be needed for more diversity in their recommendations.

The distinction between offline and online recommender systems is something not considered in the recommender system design by this thesis, but it's an important one. Instead of generating recommendations on a fixed schedule, like once a day, it is desirable to generate recommendations in real time. The way Netflix does it is to use faster and simpler algorithms for real time recommendations and more advanced supervised learning algorithms for offline recommendations which are more computationally complex.
%As the complexity of the system grows the line between offline and online algorithms become more blurred and ...

\subsection{System}

In the future the functionality of the exporter module could be replaced by either the remote API or the admin web interface.

The filtering functionality in the reader module could be moved to an external tool operating directly on the database. This would allow for a more efficient input of the interaction history data by streaming it directly into the database. The current implementation loads it all into memory and thereafter populates the database, which might cause memory problems if large interaction history is given.

Extensions to allow for metadata about the interactions could be beneficial. Filtering away very old interactions or the ability to split the training and test sets depending on the interaction date can be very useful.


\section{Datasets}

The usage of the \textit{movielens1m} dataset can be questioned as the dataset is made up by ratings and then converted to unweighted binary form. This introduces a lot of noise and the question is what conclusions can be drawn from such a dataset.

The optimal parameters for \textit{eswc2015books} are fairly strange. For \textit{katz-eig} $K = 1$ gives the best recommendation quality, but that means the recommendations are given by a rank-1 approximation. Also for \textit{link-analysis} $\gamma = 0$ gives the best recommendation quality. Which means that no penalizing users with many purchases occur.


\section{Evaluation}

The evaluation metric of \textit{F-measure} can be questioned/discussed...? Why no cost function? Probably would be better (if could be found, but can it?) ?

Alternative evaluation metrics. \rmse, rank, accuracy, diversity.

A common deficiency for evaluation metrics is the lack of formalization. The metrics themselves are well defined but implementation details differ and are sometimes missing which can lead to different results between similar experiments.


\section{Parameter analysis}

\subsection{katz-eig}

\subsection{link-analysis}

Changes of $\eta$ as long as $\eta > 0$ appears to not have a very big impact. This can be explained as the role of $\eta$ is to keep the user representativeness score high for each user to itself. The normalization step in the nextcoming iteration removes the impact of larger value of $\eta$.

$\eta < 0 $ is harder to explain...

Referenced article introduces $\eta = 1$, but treats it as a constant. No further comments about it!

Same article: $\gamma$ is said to be in the range 0 to 1, but I found that larger ones are good and often even better.


\section{Parameter tuning}

Optimization technique interesting to try is bayesian optimization!

Clustering would be nice to explore.

The lack of an easy cost function to optimize against is a hinderance. In practice calculating the top-10 recommendations is a slow process and a faster way to evaluate the recommendation quality given a set of parameters would give a nice speed improvement.


\section{Algorithm comparisons}

\textit{katz-eig} has much better runtime performance compared to \textit{link-analysis} and generates comparable recommendation quality. \textit{link-analysis} have slightly better quality for sparse datasets, which is supported by the literature. In fact the primary motivation for \textit{link-analysis} is to generate recommendations for sparse data. But still, \textit{katz-eig} gives comparable performance.

Optimization is very expensive, but generating recommendations for the full datasets for both algorithms are fairly fast. They can both be useful commercially when run offline.


\section{All the rest...}

\begin{enumerate}
    \item ??
\end{enumerate}

\chapter{Results}\label{cha:Results}

Pure facts and as objectively as possible. Don't analyze, comment or evaluate.

Use same sections as in methodology?

If describe implementation process, only describe largest decisions.

\section{Prestudy}\label{sec:prestudy}

Resulting graphs from LinkAnalysis. Use different kinds of datasets.

\begin{itemize}
    \item Graph over $\gamma$ with fixed $\eta$.
    \item Graph over $\eta$ with fixed $\gamma$.
    \item Graph over $\gamma$, $\eta$.
    \item Graph over $T$, see that things flattens out.
\end{itemize}

Similarly for KatzEig.

\section{Implementation}\label{sec:implementation}

Compare custom\_neighbour, custom\_neighbour\_step, fminsearch, ... with each other. Compare runtime and accuracy over different kinds of datasets.

Similarly for KatzEig.

Also: database considerations. Python plugin system.

\section{Evaluation}\label{sec:evaluation}

??

%\begin{chapter-appendix}
%
%\section{Ett par långa bevis}
%
%Det här är en appendix-del av det aktuella kapitlet.
%
%\end{chapter-appendix}

\chapter{Methodology}\label{cha:method}

\textit{Describe how the work is supposed to be done. Should be able to replicate the work.}

\textit{Discuss implementation choices and alternatives here?}

\Warning[TODO]{ section: Model? }

\section{Development methodology}

The software development side of the thesis

\textit{Agile, iterative etc...}

The software will be developed using agile inspired methods. Iterative development will be used to produce a simple prototype and then iteratively improve and add more features. The early priority is to produce a working chain from reading data to storing recommendations in the database.

Small incremental goals will be used, an example goal could be to complete a reader plugin for a specific dataset.

Automatic tests shall be used but not in the test driven development way.


\subsection{Programming languages}

Why matlab? Why python? Why not something else? (Julia, numpy, scipy, C, C++, ...)?

The existing algorithms exists in a prototype form in Matlab. The thesis will continue to use the algorithms written in Matlab for easy prototyping and modifications. Python will be used as glue and be used to implement all modules (see \sectionref{sec:sysoverview}).

Using other languages or platforms, such as Julia, C, C++, or Python with NumPy or SciPy could give performance improvements, but it's not in the scope of this thesis. Also as Comordo is in it's startup phase with focus on prototyping it is valuable to continue with a platform familiar to them.



\section{Data}

The available datasets are described in detail in \appendixref{cha:datasets}. All data will in unweighted binary form \eqref{eq:hist}.

An alternative to unweighted binary data is weighted data, where $h_{u, i} = x$ means that user $u$ has interacted with item $i$ $x$ times. There is some support in the available datasets (\textit{alpha}, \textit{alpha2}) but most of the datasets does not, which is why the focus is on datasets in unweighted binary form.

Another popular format is ratings, which \textit{movielens1m} is derived from. Generating recommendations with explicit feedback, such as ratings, is well researched but fundamentally different from implicit feedback systems. The focus on this thesis is recommendations for implicit feedback systems which is why ratings are not considered in their raw form.

During supervised learning the datasets will be divided into training, validation and test sets with a ratio of 70\%, 15\% and 15\% respectively. When a validation set is not necessary, it will be ignored and only the training and test sets will be used. Alternatively a new split with only training and test sets could be used with, for example, a split of 80\% and 20\% could be used. This thesis uses the same approach because of simplicity, there's no need to track different kinds of splits for the same data.

As mentioned in \sectionref{sec:background:theory:suplearn} there are different ratios commonly used to split datasets. There is no ratio which is always the best, they depend on the amount of data available, the modeled domain and the algorithms chosen. A split of 70/15/15 was chosen early for simplicity reasons.



\subsection{Evaluation}\label{sec:background:theory:eval}

A common technique to evaluate the quality of recommendations as sets is with \textit{Precision}, \textit{Recall} and \textit{F-measure}
\footnote{The 2nd Linked Open Data-enabled Recommender Systems Challenge uses \textit{F-measure}, 2015. \url{http://sisinflab.poliba.it/events/lod-recsys-challenge-2015/}}
. \citep{bobadilla2013recommender}

\Warning[TODO]{ Here $h_{u, i}$ isn't the interaction history, but the test/validation set }

First, \textit{true positives}, $TP$, is the sum of all correctly predicted positive samples.

\begin{equation} \label{eq:tp}
    TP = \sum_{u, i} r_{u, i} = 1 \land \, h_{u, i} = 1
\end{equation}

Conversely \textit{false positives}, $FP$, is the sum of all falsely predicted positive samples.

\begin{equation} \label{eq:fp}
    FP = \sum_{u, i} r_{u, i} = 1 \land \, h_{u, i} = 0
\end{equation}

And \textit{false negatives}, $FN$, is the sum of all falsely predicted negative samples.

\begin{equation} \label{eq:fn}
    FN = \sum_{u, i} r_{u, i} = 0 \land \, h_{u, i} = 1
\end{equation}

Then \textit{Precision} $P$ and \textit{Recall} $R$ is defined as

\begin{equation} \label{eq:precision}
    P = \TP / (\TP + \FP)
\end{equation}

\begin{equation} \label{eq:recall}
    R = \TP / (\TP + \FN)
\end{equation}

Loosely \textit{Precision} signifies how well the recommended items correspond to the users' preferences and \textit{Recall} signifies how well the users' preferences fits with the recommendations.

In many ways precision and recall are competing measures, when optimizing for precision recall decreases and vice versa.  As the number of recommendations $N$ grow precision is expected to be lower and recall is expected to be higher. \citep{hu2008collaborative}


\textit{F-measure}, $\mathit{F1}$, is defined as the harmonic mean of precision and recall \eqref{eq:f1} as a combined measure of precision and recall.

\begin{equation} \label{eq:f1}
    \mathit{F1} = 2 * P * R / (P + R)
\end{equation}

%\Warning[TODO]{ Evaluations for top-N }

% TODO remove?
Another evaluation method commonly used to evaluate explicit feedback system with ratings is the Root of Mean Square Error \rmse \hspace{0.2ex} \eqref{eq:rmse} where $\mathit{pval}$ is the predicted ratings for the user-item interaction matrix $h$.
\citep{bobadilla2013recommender}
%\Warning[TODO]{ Conflicting variable names? }

\begin{equation} \label{eq:rmse}
    \mathtt{RMSE} = \sqrt{\mathbf{E}\left\{ (\mathit{pval} - h)^2 \right\}}
\end{equation}
\Warning[TODO]{ Matrices are usually big lettered! }






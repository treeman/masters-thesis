
\section{Training Curves}\label{sec:graphs:training_curves}

Both \textit{katz-eig} and \textit{link-analysis} are iterative algorithms. Their descriptions use the phrase "repeat until convergence".

With training curves, which plots the evaluation metric with the respect to epochs, the number of iterations, the effect of running more iterations can be seen.

\Warning[TODO]{ Explain evaluations! }

\subsection{katz-eig}

%\FloatBarrier

%\begin{figure}[h!]
  %\centering
    %\includegraphics[width=0.8\textwidth]{fig/katzeig_t/alphaS_katzeig_t.png}
    %\caption{\textit{alphaS}}
%\end{figure}

%\begin{figure}[h!]
  %\centering
    %\includegraphics[width=0.8\textwidth]{fig/katzeig_t/eswc2015books_katzeig_t.png}
    %\caption{\textit{eswc2015books}}
%\end{figure}

%\begin{figure}[h!]
  %\centering
    %\includegraphics[width=0.8\textwidth]{fig/katzeig_t/eswc2015movies_katzeig_t.png}
    %\caption{\textit{eswc2015movies}}
%\end{figure}

%\begin{figure}[h!]
  %\centering
    %\includegraphics[width=0.8\textwidth]{fig/katzeig_t/movielens_katzeig_t.png}
    %\caption{\textit{movielens1m}}
%\end{figure}

%\begin{figure}[h!]
  %\centering
    %\includegraphics[width=0.8\textwidth]{fig/katzeig_t/romeo_katzeig_t.png}
    %\caption{\textit{romeo}}
%\end{figure}

%\FloatBarrier


\subsection{link-analysis}

%\begin{itemize}
    %\item alpha
    %\item alpha2
    %\item eswc2015movies
    %\item eswc2015music
    %\item eswc2015books
    %\item movielens1m
    %\item romeo
%\end{itemize}




\subsection{link-analysis}

There are two parameters to \textit{link-analysis}: $\gamma$ and $\eta$, they are both continuous. At $\eta = 0$ all recommendations will always be 0.

The following plot is evaluated using \textit{F-measure}  over the parameter space of both $\gamma$ and $\eta$.

\begin{figure}[h!]
    \includegraphics[width=\textwidth]{fig/link_eta_gamma/alphaS_link.png}
    \caption{\textit{alphaS}}
    \label{fig:linkalphaS}
\end{figure}

For \textit{alphaS} it appears $\gamma < 0$ is a very bad choice. Both $\eta > 0$ and $\eta < 0$ seems to be fair choices, with $\eta > 0$ being slightly better. As long as $\eta > 0$, the specific choice of $\eta$ doesn't seem to matter that much. Curiously $\gamma = 0$ represents a peak. There's an anomaly at $\eta = -2$ which is markedly worse than $\eta = -1$ and $\eta = -3$.

\FloatBarrier

\begin{figure}[h!]
\centering
\begin{minipage}{.5\textwidth}
    \centering
    \includegraphics[width=\linewidth]{fig/link_eta_gamma/movielens_link.png}
    \captionof{figure}{\textit{movielens1m}}
\end{minipage}%
\begin{minipage}{.5\textwidth}
    \centering
    \includegraphics[width=\linewidth]{fig/link_eta_gamma/romeo_link.png}
    \captionof{figure}{\textit{romeo}}
\end{minipage}
\end{figure}

\FloatBarrier

Similarly for \textit{movielens1m} and \textit{romeo}, $\gamma > 0$ is generally better than $\gamma < 0$. This time $\gamma = 0$ does not represent the maximum function value. The function space seems to be fairly smooth, almost convex, except for $\eta = 0$. For \textit{movielens1m} $\eta < 0$ is the better choice.

\FloatBarrier

\begin{figure}[h!]
    \includegraphics[width=\textwidth]{fig/link_eta_gamma/eswc2015books_link.png}
    \caption{\textit{eswc2015books}}
\end{figure}

\FloatBarrier

A closer look at \textit{eswc2015books} reveals that the function space isn't as smooth as it might have seemed in the previous plots, several local optima can be seen.
%A decline around $\eta = -2$ is similar to that of \textit{alphaS}. Another similarity is that as long as $\eta > 0$, the exact value of $\eta$ doesn't seem to matter that much.

\newpage

Interestingly some of the datasets have their maximum at $\eta > 0$ but others have $\eta < 0$. What follows is some plots over $\eta$ and a fixed $\gamma = 1$.

\FloatBarrier

\begin{figure}[h!]
\centering
\begin{minipage}{.5\textwidth}
    \centering
    \includegraphics[width=\linewidth]{fig/link_eta/alphaS_link_eta.png}
    \captionof{figure}{\textit{alphaS}}
\end{minipage}%
\begin{minipage}{.5\textwidth}
    \centering
    \includegraphics[width=\linewidth]{fig/link_eta/eswc2015books_link_eta.png}
    \captionof{figure}{\textit{eswc2015books}}
\end{minipage}
\end{figure}

\begin{figure}[h!]
\centering
\begin{minipage}{.5\textwidth}
    \centering
    \includegraphics[width=\linewidth]{fig/link_eta/movielens_link_eta.png}
    \captionof{figure}{\textit{movielens1m}}
\end{minipage}%
\begin{minipage}{.5\textwidth}
    \centering
    \includegraphics[width=\linewidth]{fig/link_eta/romeo_link_eta.png}
    \captionof{figure}{\textit{romeo}}
\end{minipage}
\end{figure}

\FloatBarrier

Here \textit{alphaS} and \textit{eswc2015books} have maximum \textit{F-measure} with $\eta > 0$ but \textit{movielens1m} and \textit{romeo} with $\eta < 0$.

%The actual optima, given optimization criteria of $\eta$ and $\gamma$, can be different. As can be seen in \figureref{fig:linkalphaS} the optima uses a smaller $\gamma$ and a positive $\eta$. But the local optima can still be found with either $\eta > 0$ or $\eta < 0$.

\newpage

$\gamma$ is the parameter which seems to vary the function value the most. The following plots varies $\gamma$ while holding $\eta = 1$ constant.

\FloatBarrier

\begin{figure}[h!]
\centering
\begin{minipage}{.5\textwidth}
    \centering
    \includegraphics[width=\linewidth]{fig/link_gamma/alphaS_link_gamma.png}
    \captionof{figure}{\textit{alphaS}}
\end{minipage}%
\begin{minipage}{.5\textwidth}
    \centering
    \includegraphics[width=\linewidth]{fig/link_gamma/eswc2015books_link_gamma.png}
    \captionof{figure}{\textit{eswc2015books}}
\end{minipage}
\end{figure}

\begin{figure}[h!]
\centering
\begin{minipage}{.5\textwidth}
    \centering
    \includegraphics[width=\linewidth]{fig/link_gamma/movielens_link_gamma.png}
    \captionof{figure}{\textit{movielens1m}}
\end{minipage}%
\begin{minipage}{.5\textwidth}
    \centering
    \includegraphics[width=\linewidth]{fig/link_gamma/romeo_link_gamma.png}
    \captionof{figure}{\textit{romeo}}
\end{minipage}
\end{figure}

\FloatBarrier

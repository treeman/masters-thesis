\chapter{Related Work}\label{cha:relwork}

\textit{Need to anchor problem formulation here!}

\textit{Practically complete until halftime eval!}

Netflix has some comments on their recommender system.
\url{http://techblog.netflix.com/2012/06/netflix-recommendations-beyond-5-stars.html}
\url{http://techblog.netflix.com/2012/04/netflix-recommendations-beyond-5-stars.html}

Niklas' master thesis \citep{niklas}is concerned with recommending with clustering. Not concerned with runtime. Also has a bunch of relevant definitions.

The link-analysis algorithm is introduced by \citep{huang2004link} and used by \citep{huang2007comparison}. They do not do an in-depth analyze of different values for $\eta$ nor $\gamma$. They evaluate the different algorithms using \textit{precision}, \textit{recall}, \textit{F-measure} and \textit{rank score}.

Recommender systems survey \citep{bobadilla2013recommender} is a summary of different recommender systems approaches.

Clustering has been commented on by \citep{cacheda2011comparison}.

$h_{u, i} = 0.7$ could mean that user $u$ has watched 70\% of the movie $i$, in the context of movie watching. \citep{hu2008collaborative}. Defining \textit{implicit feedback} systems.

Top-N recommendation problem is the real problem of many on-line recommender systems \citep{lai2012hybrid}.
\Warning[TODO]{ Remove? Keep? Doesn't add much else. }

Many approaches combines different algorithms. See netflix, \citep{lai2012hybrid}, ...
\Warning[TODO]{ More! }

katz-eig used in \citep{shin2012multi}. Not sure what it adds? Described by \citep{liben2007link}, introduced by \citep{katz1953new} (not very useful though).

\citep{bennett2007netflix} is a summary of the netflix prize. Intro?

ESWC 2014 discussions by \citep{di2014linked}, \citep{heitmann2014semstim}, \citep{ostuni2014linked}.

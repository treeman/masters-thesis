
\section{Motivation}\label{sec:intro:motivation}

Being able to make choices, of any kind, has always been an important skill and perhaps it's more important now than ever before. It's hard to choose what products to buy, what music to listen to, what posts to read and what videos to watch as there are so many choices but a limited amount of time. In Youtube alone over 300 hours of video is uploaded every minute \footnote{Youtube Statistics, 2015. \url{http://www.youtube.com/yt/press/statistics.html}}.

This is why content providers and e-commerce are using recommendations, where products believed to appeal to the consumer are presented more prominently on the sites. Recommendations have become an important part of their business and companies such as Netflix are investing heavily into making their recommendations better \footnote{ Netflix: Recommendations beyond 5 stars (Part 1), 2012. \url{http://techblog.netflix.com/2012/04/netflix-recommendations-beyond-5-stars.html} } \footnote{ Netflix: Recommendations beyond 5 stars (Part 2), 2012. \url{http://techblog.netflix.com/2012/06/netflix-recommendations-beyond-5-stars.html} }.

A common practice among e-commerce is to produce related recommendations where products are linked to related, similar, products. Another type is personal recommendations where products are recommended specifically for a single user given their interaction history.

There are constant algorithms which produce these recommendations, they are quite fast and simple but algorithms based on machine learning can produce more relevant recommendations. They work by learning from the data and building a model used to make predictions. The drawback is performance and complexity, they are more demanding than the constant algorithms.

\textit{Explicit feedback} recommender systems, which are concerned with ratings or other voluntary user feedback, have been researched extensively but \textit{implicit feedback}, which passively collect information about the user, are not as extensively researched. \citep{hu2008collaborative}

\Warning[TODO]{ Expand on it more. Comordo focuses on implicit feedback which is why we are doing this... }

The question then becomes how can we make high quality recommendations using learning algorithms in practice using real data in a reasonable amount of time?


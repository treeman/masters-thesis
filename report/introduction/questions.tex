
\section{Questions}\label{sec:intro:questions}

\textit{2-4 related questions. The more specific the better!}

\begin{itemize}

    \item How can a recommender system be designed to allow for easily extendible input- and output handling?


    \item How can learning and recommendation using \textit{link-analysis} and \textit{katz-eig} be performed in practice, with regards to speed and recommendation quality?

        \begin{itemize}
            \item How shall learning and optimization of their parameters be done?
        \end{itemize}

        To answer that question, an exploration of the function space of the parameters with regards to the evaluation criteria might be necessary. Then additional questions might need answers:

        \textit{These could possibly be changed during thesis writing}

        \begin{itemize}
            \item Is the function convex or concave?
            \item What search optimization techniques are applicable?
            \item Are there large differences in value across the function space?
            \item In what range is it reasonable to expect the optimal parameters to be found in?
        \end{itemize}


    \item Is clustering beneficial for speed or recommendation quality, with respect to \textit{link-analysis} and \textit{katz-eig}?

        The hypothesis is improvement in both aspects and it may allow handling of larger datasets.

        \begin{itemize}
            \item Is it possible to utilize parallelization in conjunction with clustering?
        \end{itemize}

        Clustering groups related users and items together to produce smaller sets of history data and recommendation data could be generated for them independently. Afterwards they can be combined to generate recommendations for the initial larger dataset.

    \Warning[QUEST]{ Is this necessary or can I skip this question? }

\end{itemize}



\section{Description of the datasets}\label{sec:datasets}

These are the datasets used by the thesis, a summary of the available datasets and their size is given by \tableref{tab:datasets}. Some of the datasets (\textit{alpha}, \textit{alpha2}, \textit{alphaS}, \textit{romeo}) are given by some of Comordo's clients and they do not want the data to be publicly available. Instead a high level description of the datasets are given.

All of the datasets will be in unweighted binary form \eqref{eq:hist}. Some of the datasets (\textit{alpha}, \textit{alpha2}, \textit{alphaS}) support weighted form \eqref{eq:whist} but the other datasets does not, so they are transformed into unweighted binary form. Another given format is ratings, which \textit{movielens1m} use. Generating recommendations with explicit feedback, such as ratings, is well researched but fundamentally different from implicit feedback systems. The focus of this thesis is on implicit feedback systems which is why ratings are not considered in their raw form, they are converted to unweighted binary form.

During supervised learning the datasets will be divided into training, validation and test sets with a ratio of 70\%, 15\% and 15\% respectively. The split is done by randomly distributing all items in the interaction history and distributing amongst the sets. In a matrix representation it can be thought as randomly assigning each non-zero value from the interaction matrix $A$ to either $A_{train}$, $A_{val}$ or $A_{test}$ while keeping all other elements as zero.

When a validation set is not necessary, it will be ignored and only the training and test sets will be used. This is done for simplicity and to reduce the number of different datasets needed to keep track of.  As mentioned in \sectionref{sec:background:suplearn} there are different ratios commonly used to split datasets. There is no ratio which is always the best, they depend on the amount of data available, the modeled domain and the algorithms chosen. A split of 70/15/15 is chosen early for simplicity reasons.

\begin{table}[h]
    \centering
    \begin{tabular}{| c | r | r | r | r | l |}
        \hline
        \textbf{dataset}        & \textbf{users}    & \textbf{items}    & \textbf{elements} & \textbf{sparsity}  \\ \hline

        \textit{alpha}          &   100002          & 219767            & 904201            & 0.0041\%           \\ \hline
        \textit{alpha2}         &   75007           & 345674            & 1945115           & 0.0075\%           \\ \hline
        \textit{alphaS}         &   16444           & 5000              & 26035             & 0.0316\%           \\ \hline
        \textit{eswc2015books}  &   1398            & 2609              & 11600             & 0.32\%             \\ \hline
        \textit{eswc2015movies} &   32169           & 5389              & 638268            & 0.37\%             \\ \hline
        \textit{eswc2015music}  &   52072           & 6372              & 1093851           & 0.33\%             \\ \hline
        \textit{movielens1m}    &   6040            & 3706              & 1000209           & 4.5\%              \\ \hline
        \textit{romeo}          &   8321            & 722               & 205534            & 3.4\%              \\ \hline

    \end{tabular}
    \caption{A summary of the used datasets}
    \label{tab:datasets}
\end{table}

\FloatBarrier

What follows is a description of each available dataset and where they can be found, if applicable.

\begin{description}
    \item[alpha, alpha2, alphaS] \hfill

        Anonymous datasets representing purchase history provided by an e-commerce client. The dataset is given in a weighted form \eqref{eq:whist} but is converted to unweighted binary form \eqref{eq:hist}.

        \textit{alpha} is a randomly sampled dataset. It contains 100002 users, 219767 items with 904201 interactions.

        \textit{alpha2} is another randomly sampled dataset, independently sampled from \textit{alpha}, filtered to only contain users with $\geq 2$ purchases. It contains 75007 users, 345674 items with 1945115 interactions.

        \textit{alphaS} is a subset of \textit{alpha2}. It contains 16444 users, 5000 items with 26035 interactions. This is often used as \textit{alpha} and \textit{alpha2} are very large and the runtime is very long.

        % >= 2
        %Found 75007 users and 345674 items
        %75007 users, 345674 items
        %1945115 interactions, 0.0075\% sparsity

        %Found 100002 users and 219767 items
        %100002 users, 219767 items
        %904201 interactions, 0.0041\% sparsity


    \item[eswc2015movies, eswc2015music, eswc2015books] \hfill

        These are the datasets used in the 2nd Linked Open Data-enabled Recommender Systems Challenge
        \footnote{2nd Linked Open Data-enabled Recommender Systems Challenge, 2015. \url{http://sisinflab.poliba.it/events/lod-recsys-challenge-2015/}}.
        The data have been collected from Facebook profiles about personal preferences, ''likes``, for movies, books and music.
        \footnote{DataSet | 2nd Linked Open Data-enabled Recommender Systems Challenge, 2015. \url{http://sisinflab.poliba.it/events/lod-recsys-challenge-2015/dataset/}}.

        The datasets were originally split into training sets and evaluation sets. The evaluation sets does not contain any user-product mappings and for evaluation purposes this thesis will only concern itself with the training set part of the datasets.

        \textit{eswc2015books} contains 1398 users with 11600 likes for 2609 items. The dataset contains likes for books, characters, genres and writers.

        \textit{eswc2015movies} contains 32159 users with 638268 likes for 6389 items. The dataset contains likes for movies, actors, directors, characters and genres.

        \textit{eswc2015music} contains 52072 users with 1093851 likes for 6372 items. The dataset contains likes for albums, artists, bands, compositions and genres.

        For the purpose of this thesis, the different item types are treated as a single type. For example no care is taken to cross-reference liked genres with movies in that genre. The only thing considered is the unweighted binary user-item interaction history.

        %Movies
        %Found 32159 users and 5389 items
        %32159 users, 5389 items
        %638268 interactions, 0.37\% sparsity

        %Music
        %Found 52072 users and 6372 items
        %52072 users, 6372 items
        %1093851 interactions, 0.33\% sparsity

        %Books
        %Found 1398 users and 2609 items
        %1398 users, 2609 items
        %11600 interactions, 0.32\% sparsity


    \item[movielens1m] \hfill

        The MovieLens 1M dataset \footnote{Grouplens: MovieLens dataset, 2015. \url{http://grouplens.org/datasets/movielens/}} is a collection of ratings (1-5) taken from the MovieLens website \footnote{MovieLens homepage. \url{https://movielens.org/}}.

        Ratings are transformed to unweighted binary form using \eqref{eq:rating2binary}.

        This is by no means a perfect transformation as a rating of 1 means the user has consumed an item but didn't enjoy it, while our model only concerns itself with interactions. Noise is introduced into the dataset and the recommendations loose relevance with respect to the original unmodified dataset. It is still possible to evaluate recommendation using \textit{F-measure} with respect to the new dataset in unweighted binary form, but no relevant conclusions can be made for the users themselves.

        The dataset contains 6040 users with 1000209 ratings for 3706 movies.

        %Found 6040 users and 3883 items
        %Filtered down to 6040 users and 3706 items
        %6040 users, 3706 items
        %1000209 interactions, 4.5\% sparsity


    \item[romeo] \hfill

        An anonymous dataset representing purchase history provided by an e-commerce client. The dataset is in unweighted binary form.

        The dataset contains 8321 users, 722 items and 205534 interactions.

        %Found 8321 users and 722 items
        %8321 users, 722 items
        %205534 interactions, 3.4\% sparsity


\end{description}

Some of the datasets are very large and later in the thesis all datasets are not used as the runtime is so long. For example not all datasets are handled in the parameter analysis in \sectionref{sec:parameters}. Specifically the datasets \textit{alpha}, \textit{alpha2} and \textit{eswc2015music} are often excluded and \textit{eswc2015movies} is also excluded sometimes.


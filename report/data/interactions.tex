
\section{Number of interactions}\label{sec:result:interactions}

What follows is some plots describing the number of interactions each user has and the number of interaction each item has in the datasets. It's useful for identifying outliers and possibly for identifying defining features of a dataset.

The plot on the left describe how many users have a fixed number of item interactions and conversely the plot on the right describe how many items have a set number of user interactions. The histograms are also represented in logarithmic scale.

\FloatBarrier

\twopic{fig/data/alphaS_items_per_user.png}{fig/data/alphaS_users_per_item.png}{
\textit{alphaS}
}

In \textit{alphaS} each user and each item has interactions with a small fraction of the available items and users. There are many users with very few interactions and also many items which few users has interacted with. There are no users or items without any interactions, but there are 11923 out of 16444 users and 2588 items out of 5000 with with only one interaction. This can be compared to the 26035 total interactions in the dataset. There are some users with more than 400 interactions and some items which have interacted with more 600 users.

\newpage

\twopic{fig/data/eswc2015books_items_per_user.png}{fig/data/eswc2015books_users_per_item.png}{
\textit{eswc2015books}
}

\twopic{fig/data/eswc2015movies_items_per_user.png}{fig/data/eswc2015movies_users_per_item.png}{
\textit{eswc2015movies}
}

\twopic{fig/data/eswc2015music_items_per_user.png}{fig/data/eswc2015music_users_per_item.png}{
\textit{eswc2015music}
}

\FloatBarrier

All \textit{eswc} datasets have similar distributions with more concentrated interactions. There is a lower limit for the number of interactions each user has, this is probably a constraint used when the datasets were made.
There are also no extreme user outliers with many more interactions than the norm.

The item interactions are more spread, with many items having interacted with relatively few user but some items having a lot of interactions. \textit{eswc2015books} have 1134 out of 2609 items with only one user interaction. \textit{eswc2015movies} and \textit{eswc2015music} in comparison have 2 out of 5389 items and 1 out of 6372 items with one user interaction.

%\newpage

\twopic{fig/data/movielens1m_items_per_user.png}{fig/data/movielens1m_users_per_item.png}{
\textit{movielens1m}
}

\twopic{fig/data/romeo_items_per_user.png}{fig/data/romeo_users_per_item.png}{
\textit{romeo}
}

Both \textit{movielens1m} and \textit{romeo} have a more normalized look to them, especially with the number of user interactions per item compared to the other datasets. There are still outliers with many more interactions however. There are no users with less than 2 item interactions and there are no items without a user interaction. 114 out of 3706 items and 17 of 722 items have 1 user interaction in \textit{movielens1m} and \textit{romeo} respectively.

\FloatBarrier

In general two distinct types of users can be identified. The first is a user with only a couple of item interactions, this appears to be the most common type of user. It could possibly be users who try out a service but for some reason they do not continue or they are new users who just recently started using the service. The other user type is the one with a lot of item interactions, way more than the norm, and they are quite rare
\footnote{Parallels can be drawn to what is known as big spenders or ``whales'' in the social-gaming community. They make up a tiny group of the community but they drive most of the revenue for the game publishers. For a more in depth discussion see \\
VentureBeat: What it means to be a ``whale'' — and why social gamers are just gamers, 2013. \\
\url{http://venturebeat.com/2013/03/14/whales-and-why-social-gamers-are-just-gamers/} }
. They do not exist in the \textit{eswc} datasets.

A similar classification can be made for items. The vast majority of items has only a couple of user interactions. Perhaps these are new items few users have found out about or niche items not interesting to most users.
A large fraction of the items in \textit{alphaS} and \textit{eswc2015books} have only one interaction (51\% and 43\%).  Then there are items with a lot more user interactions than what is common.

The following plots display the number of the most popular items and the number of users they collectively interact with. This is useful for investigating how top heavy the datasets are. The dashed lines represents the number of items required to include 95\% of all users, a summary of the required number of items can be found in \tableref{tab:top_data}.

\begin{table}[h!]
    \centering
    \begin{tabular}{| c | r | r | r | r | r | l |}
        \hline
        \textbf{dataset}        & \textbf{items needed}  & \textbf{items total} & \textbf{item ratio}  \\ \hline

        \textit{alphaS}         &   3058          & 5000 & 61\%              \\ \hline
        \textit{eswc2015books}  &   120           & 2609 & 4.6\%                \\ \hline
        \textit{eswc2015movies} &   55            & 5389 & 1.0\%               \\ \hline
        \textit{eswc2015music}  &   78            & 6372 & 1.2\%               \\ \hline
        \textit{movielens1m}    &   13            & 3706 & 0.35\%               \\ \hline
        \textit{romeo}          &   21            & 722  & 2.9\%               \\ \hline

    \end{tabular}
    \caption{This table describes how many of the most used items are necessary to include in a set so 95\% of all users have interacted with the set.}
    \label{tab:top_data}
\end{table}

\twodiffpic{fig/data/alphaS_top_data.png}
{\textit{alphaS}. 3058 of 5000 (61\%) of the items are necessary to include 95\% of all users.}
{fig/data/eswc2015books_top_data.png}
{\textit{eswc2015books}. 120 of 2609 (4.6\%) of the items are necessary to include 95\% of all users.}

\twodiffpic{fig/data/eswc2015movies_top_data.png}
{\textit{eswc2015movies}. 55 of 5389 (1.0\%) of the items are necessary to include 95\% of all users.}
{fig/data/eswc2015music_top_data.png}
{\textit{eswc2015music}. 78 of 6372 (1.2\%) of the items are necessary to include 95\% of all users.}

\twodiffpic{fig/data/movielens1m_top_data.png}
{\textit{movielens1m}. 13 of 3706 (0.35\%) of the items are necessary to include 95\% of all users.}
{fig/data/romeo_top_data.png}
{\textit{romeo}. 21 of 722 (2.9\%) of the items are necessary to include 95\% of all users.}

%alphaS: 95\% of all users concentrated on 3058 of the items
%eswc2015books: 95\% of all users concentrated on 120 of the items (2609 tot, 4.6\%)
%eswc2015movies: 95\% of all users concentrated on 55 of the items (5389 tot, 1.02\%)
%eswc2015music: 95\% of all users concentrated on 78 of the items (6372 tot, 1.22\%)
%movielens1m: 95\% of all users concentrated on 13 of the items (3706 tot, 0.35\%)
%romeo: 95\% of all users concentrated on 21 of the items (722 tot, 2.91\%)

Of the different datasets, \textit{alphaS} is a clear outlier. It is nowhere near as top heavy as the other datasets are, requiring over 60\% of all items to reach 95\% of the users. This can in part be explained by the large number of users with only one interaction, 11923 out of 16444 users or in other words 72\% of all users have only one item interaction.

In contrast \textit{eswc2015books} require 4.6\% of the items and \textit{romeo} require 2.9\% of the items, which means few of the popular items are required to include most of the users. The other datasets are even more top heavy with \textit{eswc2015movies} and \textit{eswc2015music} only require 1.0\% and 1.2\% of the items. For \textit{movielens1m} only 0.35\%, namely 13, of the items are needed. In order words this means that 95\% of all users in the dataset has seen at least one movie from the 13 most watched movies.

This phenomena where very few of the most popular items command the attention of most of the user base is also seen in mobile app stores where 1.6\% of app developers make more than the other 98.4\% combined
\footnote{
readwrite: Among Mobile App Developers, The Middle Class Has Disappeared, 2014. \\
\url{http://readwrite.com/2014/07/22/app-developers-middle-class-opportunities}
}.

\FloatBarrier

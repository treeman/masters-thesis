\chapter{Conclusions}\label{cha:conclusions}

A recommender system with the recommendation algorithms \textit{katz-eig} and \textit{link-analysis} was built and optimization strategies for learning the algorithms' parameters and fit them to different datasets were implemented and evaluated.

The recommender system uses a dynamic plugin system for flexible and easily extendible input handling.

The datasets were analyzed and clusters were found in all datasets. Recommendation quality was measured using \textit{F-measure} and the effect of varying the different parameters were analyzed.  Varying $\beta$ for \textit{katz-eig} has little to no effect and values of $\eta$ mostly varies depending on the sign.

The best optimization strategy for \textit{katz-eig} was to fix $\beta = \| A \|_2$ and optimize $K$ using a hill climbing algorithm. It gave similar recommendation quality compared to a full grid search but with better speed. Similarly the best optimization strategy for \textit{link-analysis} only examines $\eta = 1 \text{ or } -1$ and uses an adaptive hill climbing algorithm to search for $\gamma$.

\textit{katz-eig} gives better or comparable recommendation quality compared to \textit{link-analysis}, where \textit{link-analysis} gives slightly better performance with sparse data. \textit{katz-eig} is superior speed wise.

For future work \textit{bayesian optimization} and \textit{simulated annealing} could be explored as possibly more efficient optimization strategies.  The recommender system could be extended with regards to diversity and explaining recommendations.


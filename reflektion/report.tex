
\documentclass[11pt]{article}
\usepackage[utf8]{inputenc}
\usepackage{listingsutf8}
\usepackage{fullpage}
%\usepackage{times}
\usepackage{graphicx}
\usepackage{placeins}
%\usepackage{multicol}
%\usepackage{amsmath}
%\usepackage{tabularx}
%\usepackage{setspace}
%\usepackage{epstopdf}
%\usepackage{amassymb}
%\usepackage{latexsym}
%\usepackage[swedish]{babel}
%\usepackage{datetime}
\usepackage{hyperref}
%\usepackage{listings}
\raggedright

% Space up paragraphs a bit
\setlength{\parskip}{4mm plus1mm minus3mm}

%\linespread{1.3} % 1.5 word linespacing. \linespread{1.6} => double

\begin{document}
%\begin{titlepage}
\begin{center}

%\textsc{\LARGE University of Beer}\\[1.5cm]

%{ \Large Reflektionsdokument\\[0.5cm] }

% Title
%\HRule \\[0.4cm]
{ \LARGE Reflektionsdokument examensarbete \\[0.4cm] }

Jonas Hietala, mail@jonashietala.se, jonhi121 \\[0.2cm]

\today

\end{center}


\section*{Reflektion över hur examensarbetet relaterar till de mål som finns för programmet}

%\textit{Hur svarar ditt examensarbete mot utbildningens mål?}

%1. Matematisk grund, kan använda för att modellera
%2. Datateknik (Machine learning)
%3. Systemtänk
%4. Självständigt arbete.
%5. Muntlig/skriftlig kommunikation (english)
%6. Plan a system
%7. Develop a system
%8. Use a system
%9. Solve problems

Examensarbetet är matematiskt förankrat vilket knyter an till målet om en solid matematisk grund. Arbetet är även väl förankrat inom huvudområdet Datateknik. Planerandet och utvecklandet av ett rekommendationssystem knyter direkt an till målet om planering och utveckling av tekniska system. Arbetet har varit självständigt med problemlösning och ett systemtänk har genomsyrat hela exjobbet. Det tillverkade system används i produktion och har direkt tillfört värde till Comordo.

Rollen som ansvarig utvecklare för grunden av Comordo's rekommendationssystem uppfyller målet om att ingenjören ska vara förberedd att axla ansvarsfulla roller. Aktiv kommunikation med Comordo om krav och riktning med rekommendationssystemet visar på god verbal kommunikationsförmåga. Dokumentation av systemet samt exjobbsrapporten visar på god skriftlig kommunikation. Då all kod och dokumentation är på engelska uppfyller det kravet på kommunikation på ett främmande språk.  Presentation av examensarbetet samt demo för Comordo demonstrerar även det kommunikationsförmåga.

Inläsning på rekommendationsområdet uppfyller målet om att kunna tillägna sig ny kunskap. Experiment och hypoteser underlättar förståendet och demonstrerar målet om kunskapsbildning.

\section*{Reflektion över eget arbete}

%\begin{itemize}
    %\item Planering \\

        %Blev planeringen ett bra stöd för genomförandet? Lades det tillräckligt med tid på planeringen? Var förutsättningarna tillfredställande? Examensarbetet skall motsvara en viss arbetsinsats, (1,5 hp = en veckas heltidsarbete), hur väl överensstämmer det med ditt examensarbete?

    %\item Genomförande och rapportskrivning \\

        %Disponerades tiden på ett tillfredställande sätt? Vad var problematiskt och varför? Vad gick över förväntan och varför?
        %I de fall examensarbetet gjorts tillsammans med annan studerande beskriv hur uppdelningen av arbetet har gjorts. Beskriv om samarbetet har varit positivt eller negativt och på vilket sätt. Reflektera över hur självständigt arbetet genomfördes. Var du tillräckligt förberedd för att kunna skriva en examensarbetsrapport? Hade du tillräckliga språkkunskaper (engelska, annat språk)? Är du nöjd med hur du lyckades genomföra ditt examensarbete?
%\end{itemize}

Påbörjandet av exjobbet blev inte som jag hade tänkte mig. Jag började jobba med Comordo ganska tidigt men jag hade svårt att hitta en examinator innan jag kontaktade Fredrik som blev min examinator. Efter det dröjde det ett tag innan jag till slut fick Mattias som min handledare. Då hade jag redan jobbat hos Comordo i lite mer än 1 månad. Jag försökte göra en egen planering, men feedback kom sent och då fick vi svänga om exjobbet lite. Så planeringen var inte så bra som den borde ha varit och lite för lite tid lades på planeringen.

Arbetsinsatsen överenstämde väl med den förväntade arbetsinsatsen på 30 hp. Det krävdes ungefär lika mycket arbete som tidigare terminer. Tiden disponerades relativt bra, men det blev lite stressigt mot slutet speciellt med rapportskrivandet. Jag påbörjade rapportskrivningen ganska tidigt, men var dåligt förberedd på hur mycket arbete som faktiskt krävdes. När sedan feedback från handledare/examinator kom så var det mycket som saknades och då var tiden väldigt knapp om jag skulle hinna klart innan sommaren. Det lyckades men det var knappt om tid och mycket jobb gjorde i slutet på kort tid.

I början av exjobbet gjordes mycket av systemutvecklingen, vilket gick väldigt bra och snabbt, så i början tyckte jag att planeringen var bra. Men den tuffare teoretiska biten med läsning av artiklar och rapportskrivningen gick mycket långsammare så de blev tidsbrist mot slutet ändå. Den sena omformuleringen av exjobbets mål gjorde saken värre.

Arbetet genomfördes väldigt självständigt. Jag var väl förberedd för de tekniska utmaningarna och problemlösningen men väldigt dåligt förberedd för rapportskrivning. Det här var i stort sett den första större rapporten jag skrev på hela utbildningen och jag hade svårt med vad som förväntades och hur jag skulle skriva den. Feedback från handledaren och examinatorn var väldigt bra och hjälpte mig mycket.  Min behärskning av engelska var dock fullt tillräcklig.

Jag är generellt nöjd med mitt examensarbete även om jag hade velat hinna med lite mer undersökning av olika optimeringsstrategier. Systemutvecklingen gick över förväntan.



\section*{Reflektion över det ämnesinnehåll, kunskaper, färdigheter och förhållningssätt som var till mest nytta för examensarbetets genomförande}

%Vilka områden och kurser inom utbildningen har varit till mest nytta för examensarbetet, och vilka nya kunskaper och färdigheter har varit nödvändigt att komplettera med för att genomföra arbetet? Känner du dig väl förberedd för ditt framtida yrkesliv?

De kurser jag haft mest nytta av från min utbildning har varit linjär algebra, kombinatorisk optimering, artificiell intelligens och automatisk planering. Generellt har programmeringsfärdigheter och bekvämlighet med matematisk notation och resonerande varit väldigt nyttigt. Jag har kompletterat med en fristående online kurs via Coursera via Stanford om artificiell intelligens. Även lite extra om programmering i Matlab.

Det generellt viktigaste har varit förmågan att läsa och ta till mig vetenskapliga artiklar och sovra i det tillgängliga materialet vilket jag tränat mycket på under de åren jag har pluggat. Jag känner mig generellt väl förberedd för mitt framtida yrkesliv då jag känner mig flexibel med en stor bredd på mina kunskaper.

Viljan att bli klar innan sommaren hjälpte då mycket av rapporten ännu var kvar och tiden höll på att rinna ut. Erfarenheten från studenttiden där mycket av arbetet väldigt ofta (för ofta) sparas till slutet hjälpte då jag är van att arbeta under en tight deadline. Det är en dålig ovana att alltid spara saker till slutet men en bra förmåga att få mycket gjort när det väl gäller.

\end{document}


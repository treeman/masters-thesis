
\documentclass[11pt]{article}
\usepackage[utf8]{inputenc}
\usepackage{listingsutf8}
\usepackage{fullpage}
%\usepackage{times}
\usepackage{graphicx}
\usepackage{placeins}
%\usepackage{multicol}
%\usepackage{amsmath}
%\usepackage{tabularx}
%\usepackage{setspace}
%\usepackage{epstopdf}
%\usepackage{amassymb}
%\usepackage{latexsym}
%\usepackage[swedish]{babel}
%\usepackage{datetime}
\usepackage{hyperref}
%\usepackage{listings}
\raggedright

% Space up paragraphs a bit
\setlength{\parskip}{4mm plus1mm minus3mm}

%\linespread{1.3} % 1.5 word linespacing. \linespread{1.6} => double

\begin{document}
%\begin{titlepage}
\begin{center}

%\textsc{\LARGE University of Beer}\\[1.5cm]

%{ \Large Reflektionsdokument\\[0.5cm] }

% Title
%\HRule \\[0.4cm]
{ \LARGE Reflektionsdokument examensarbete \\[0.4cm] }

Jonas Hietala, mail@jonashietala.se, jonhi121 \\[0.2cm]

\today

\end{center}


\section*{Reflektion över hur examensarbetet relaterar till de mål som finns för programmet}

Hur svarar ditt examensarbete mot utbildningens mål?
Målen finns beskrivna i respektive programs utbildningsplan. För civilingenjörs- och högskoleingenjörsutbildning se även de mål som finns för LiTH-civilingenjören, resp. LiTH-högskoleingenjören.


\section*{Reflektion över eget arbete}

\begin{itemize}
    \item Planering \\

        Blev planeringen ett bra stöd för genomförandet? Lades det tillräckligt med tid på planeringen? Var förutsättningarna tillfredställande? Examensarbetet skall motsvara en viss arbetsinsats, (1,5 hp = en veckas heltidsarbete), hur väl överensstämmer det med ditt examensarbete?

    \item Genomförande och rapportskrivning \\

        Disponerades tiden på ett tillfredställande sätt? Vad var problematiskt och varför? Vad gick över förväntan och varför?
        I de fall examensarbetet gjorts tillsammans med annan studerande beskriv hur uppdelningen av arbetet har gjorts. Beskriv om samarbetet har varit positivt eller negativt och på vilket sätt. Reflektera över hur självständigt arbetet genomfördes. Var du tillräckligt förberedd för att kunna skriva en examensarbetsrapport? Hade du tillräckliga språkkunskaper (engelska, annat språk)? Är du nöjd med hur du lyckades genomföra ditt examensarbete? 
\end{itemize}

\section*{Reflektion över det ämnesinnehåll, kunskaper, färdigheter och förhållningssätt som var till mest nytta för examensarbetets genomförande}

Vilka områden och kurser inom utbildningen har varit till mest nytta för examensarbetet, och vilka nya kunskaper och färdigheter har varit nödvändigt att komplettera med för att genomföra arbetet? Känner du dig väl förberedd för ditt framtida yrkesliv? 

\end{document}

